\documentclass[a4paper, 11pt]{article}

\usepackage{geometry}
%\geometry{a4paper,left=30mm,right=30mm, top=20mm, bottom=20mm}
\geometry{a4paper,left=20mm,right=20mm, top=25mm, bottom=20mm}

\usepackage[ngerman]{babel}
\usepackage[utf8]{inputenc} 
\usepackage[T1]{fontenc}
\usepackage{amsmath}
\usepackage{amssymb}
\usepackage{fancyhdr}
\usepackage{graphicx}
\usepackage{tikz}
\usepackage{lscape}
\usepackage{comment}
\usetikzlibrary{positioning}

\usepackage{lastpage} % Seitenzahlen

\pagestyle{fancy}
\usepackage{mathtools}   % Lädt »amsmath« 
\newtagform{simple}{}{}{}
\usetagform{simple}

\usepackage{tabularx} %schöne tabellen
\parindent0pt %einrücken verhindern

\usepackage{polynom}
% overall sans serif font
%\renewcommand{\familydefault}{\sfdefault}
\cfoot{\thepage  \ / \pageref{LastPage}}

% % % % % % % % % % % % % % % % % % % % % % % %
% % % % % % % % % % % % % % % % % % % % % % % %
\newcommand{\modullang}{AOT}
\newcommand{\modul}{AOT MSC}
\newcommand{\blatt}{02 Übungsblatt}
\newcommand{\tutorium}{Mittwoch 12:00 Uhr}
\newcommand{\tutor}{Dr. Fricke}
\newcommand{\datum}{21. Dezember 2014}
\newcommand{\gruppe}{Gr05}
\newcommand{\RM}[1]{\MakeUppercase{\romannumeral #1}}
% % % % % % % % % % % % % % % % % % % % % % % %
% % % % % % % % % % % % % % % % % % % % % % % %

\begin{document} 

%%% Kopfzeile linker Bereich
%      gerade Seite   ungerade Seite
\lhead{\textbf{\modul}}
%%% Kopfzeile mittlerer Bereich
%      gerade Seite   ungerade Seite
\chead{\blatt}
%%% Kopfzeile linker Bereich
%      gerade Seite             ungerade Seite
\rhead{\gruppe}


	%-- Deckblatt --						      
	\title{\textbf{\modullang\\[0.25cm]}
		\normalsize{\blatt}} %Thema ändern	
	\author{\tutorium\\ \\
		Mitja Richter, 324680\\
		Tobias Pockrandt, 325550\\
		Seitenzahl: \pageref{LastPage} \\ \\ \\ \\
		%Professor: Knipping\\
		Tutor: \tutor}
	\date{Abgabedatum: \datum} %Datum ändern
	\maketitle
	\newpage
	
	%\renewcommand \thesection {\arabic{section}.}
	%\renewcommand \thesubsection {\thesection. \arabic{subsection}}
	%\renewcommand \thesubsubsection {\thesubsection. \arabic{subsubsection}}
	

\renewcommand{\labelenumi}{\alph{enumi})}
\renewcommand{\labelenumii}{(\roman{enumii})}
\renewcommand{\labelenumiii}{\arabic{enumiii}.}
%\renewcommand{\labelenumii}{\textbf{-}}	
%-- Eigentlicher Text --
\section*{1. Aufgabe - Schulze Methode\hfill {\small (5 Punkte)}}
\begin{enumerate}
\item
\begin{tabular}{c || c|c|c|c}
& zu A & zu B & zu C & zu D \\ \hline
A & & A-(15)-D-\underline{(11)}-C-(13)-B & A-(15)-D-\underline{(11)}-C & A-\underline{(15)}-D \\
B & B-\underline{(5)}-A & & B-(21)-D-\underline{(11)}-C  & B-\underline{(21)}-D \\
C & C-(13)-B-\underline{(5)}-A & C-\underline{(13)}-B & & C-\underline{(13)}-B-(21)-D \\
D & D-(17)-E-(9)-B-\underline{(5)}-A & D-\underline{(11)}-C-(13)-B & D-\underline{(11)}-C & \\
E & E-(9)-B-\underline{(5)}-A & E-\underline{(9)}-B & E-\underline{(9)}-B-(21)-D-(11)-C & E-\underline{(9)}-B-(21)


\end{tabular}\\
\begin{tabular}{c || c}
& zu E\\ \hline
A & A-\underline{(15)}-D-(17)-E\\
B  & B-(21)-D-\underline{(17)}-E\\
C & C-\underline{(13)}-B-(21)-D-(17)-E\\
D & D-\underline{(17)}-E\\
E &


\end{tabular}

\item 
\begin{align*}
p(A,B) > P(B,A) \rightarrow A \text{ besser als } B \\
p(A,C) > P(C,A) \rightarrow A \text{ besser als } C \\
p(A,D) > P(D,A) \rightarrow A \text{ besser als } D \\
p(A,E) > P(E,A) \rightarrow A \text{ besser als } E \\
\\
p(C,B) > P(B,C) \rightarrow C \text{ besser als } B \\
p(C,D) > P(D,C) \rightarrow C \text{ besser als } D \\
p(C,E) > P(E,C) \rightarrow C \text{ besser als } E \\
\\
p(B,D) > P(D,B) \rightarrow B \text{ besser als } D \\
p(B,E) > P(E,B) \rightarrow B \text{ besser als } E \\
\\
p(D,E) > P(E,D) \rightarrow D \text{ besser als } E \\
\\
\rightarrow A > C > B > D > E
\end{align*}
\end{enumerate}
\section*{2. Aufgabe - Machtverteilung im Weighted Voting\hfill {\small (5 Punkte)}}
\begin{enumerate}

\item 
$[100: 70, 60, 30, 10]$\\
- ungleiches Gewicht und Macht. Keine Diktatoren, Veto- oder Dummyspieler. Gewinnerkoalitionen:\\
$\{(P_1,P_2), (P_1,P_3), (P_1,P_2,P_3), (P_1,P_2,P_4), (P_1,P_3,P_4), (P_2,P_3,P_4), (P_1,P_2,P_3,P_4)\}$\\\\
\begin{tabular}{c | c | c}
  &  BPI & SSP\\ \hline
  $P_1$ & $6/7$ & $1/2$ \\ \hline
  $P_2$ & $3/7$ & $5/24$ \\ \hline
  $P_3$ & $3/7$ & $5/24$ \\ \hline
  $P_4$ & $1/7$ & $1/12$ \\ \hline
\end{tabular}

\item$[100: 110, 90, 5]$\\
- ungleiches Gewicht und Macht. $P_1$ ist Diktator, der Rest Dummyspieler. Gewinnerkoalitionen:\\
$\{(P_1), (P_1,P_2), (P_1,P_3), (P_1,P_2,P_3)\}$\\\\
\begin{tabular}{c | c | c}
  &  BPI & SSP\\ \hline
  $P_1$ & $1$ & $1$ \\ \hline
  $P_2$ & $0$ & $0$ \\ \hline
  $P_3$ & $0$ & $0$ \\ \hline
  $P_4$ & $0$ & $0$ \\ \hline
\end{tabular}


\item
$[100: 60, 40, 30, 20]$\\
- ungleiches Gewicht und Macht. $P_1$ ist Vetospieler. Gewinnerkoalitionen:\\
$\{(P_1,P_2), (P_1,P_2,P_3), (P_1,P_2,P_4), (P_1,P_3,P_4), (P_1,P_2,P_3,P_4)\}$\\\\
\begin{tabular}{c | c | c}
  &  BPI & SSP\\ \hline
  $P_1$ & $1$ & $3/4$ \\ \hline
  $P_2$ & $3/5$ & $1/6$ \\ \hline
  $P_3$ & $1/5$ & $1/24$ \\ \hline
  $P_4$ & $1/5$ & $1/24$ \\ \hline
\end{tabular}

\end{enumerate}
\section*{3. Aufgabe - Happy Hour in der Cocktailbar\hfill {\small (5 Punkte)}}
\begin{enumerate}
\item
Die einzige Imputation im Core wäre $(0,0,1)$ - d.h. $F$ kriegt alles und $M_1$ sowie $M_2$ nichts.
\item 
$(0.1,0.1,0.8)$ kann nicht im Core liegen, da hier $F$ mit einem der beiden $M$ ausscheren kann, um den kompletten Drink unter sich aufzuteilen - z.B. $M_1 = 0$, $M_2 =0.11$ und $F=0.89$.
\item
\begin{tabular}{c || c c c}
& $M_1$ & $M_2$ & $f$ \\ \hline
$M_1M_2F$ & 0 & 0 & 1 \\
$M_1FM_2 $ & 0 & 0 & 1 \\
$M_2M_1F$ & 0 & 0 & 1 \\
$M_2FM_1$ & 0 & 0 & 1 \\
$FM_1M_2$ & 1 & 0 & 0 \\
$FM_2M_1$ & 0 & 1 & 0 \\
SP & $\frac{1}{6}$ & $\frac{1}{6}$ & $\frac{4}{6}$ \\
\end{tabular}
\item
Für den Core gilt immer das Geschlecht, das weniger vorhanden ist, wird jeweils das komplette Getränk für sich haben können, da in jeder Situation mindestens zwei von der Gegenpartei um die Teilnahme in der 2er Koalition buhlen. Ist diese entschieden bleibt einer übrig der wiederum um eine andere Koalition buhlen muss und das Vorgehen wiederholt sich. Das bedeutet für 4M und 5F, dass alle M das Getränk bekommen und alle F nichts. Für 99M, 1F bedeutet es, dass F ein Getränk haben wird und der Rest nichts.\\

Bei dem Shapley-Value für 99M, 1F gibt es $100!$ mögliche Auszahlungsreihenfolgen. F kriegt nur keine Auszahlung, wenn Sie die erste in der Permutation ist, also in $99!$ Fällen. Das bedeutet, dass der SP für $F = 1 - \frac{99!}{100!} = \frac{99}{100}$ und da die M gleiches Gewicht besitzen dementsprechend $SP(M_i) = \frac{1}{9900}$ für $i \in [1,99]$.\\


\end{enumerate}

\section*{4. Aufgabe - Nukleolus in superadditivem Spiel\hfill {\small (5 Punkte)}}
\begin{tabular}{c || c c c}
& e(S,x) & (0,8,28) & (5,8,23) \\ \hline
A & $0-x_1$ & 0 & -5\\
B & $0-x_2$ & -8 & -8\\
C & $6-x_3$ & -22 & -17\\
AB & $6-(x_1+x_2)$ & -2 & -7\\
AC & $12-(x_1+x_3)$ & -12 & -12\\
BC & $26-(x_2+x_3)$ & -10 & -5\\

\end{tabular}\\
um die Varianz noch zu verkleinern könnte man noch:

\begin{tabular}{c || c}
& (5,10,19) \\ \hline
A &-5\\
B & -10\\
C & -13\\
AB & -9\\
AC & -8\\
BC & -5\\

\end{tabular}\\
anwenden.

\section*{5. Weighted Voting und CFG mit $\geq$ 3 Spielern\hfill {\small (10 Punkte)}}

\end{document}
