\documentclass[a4paper, 11pt]{article}

\usepackage{geometry}
%\geometry{a4paper,left=30mm,right=30mm, top=20mm, bottom=20mm}
\geometry{a4paper,left=20mm,right=20mm, top=25mm, bottom=20mm}

\usepackage[ngerman]{babel}
\usepackage[utf8]{inputenc} 
\usepackage[T1]{fontenc}
\usepackage{amsmath}
\usepackage{amssymb}
\usepackage{fancyhdr}
\usepackage{graphicx}
\usepackage{tikz}
\usepackage{lscape}
\usepackage{comment}
\usetikzlibrary{positioning}

\usepackage{lastpage} % Seitenzahlen

\pagestyle{fancy}
\usepackage{mathtools}   % Lädt »amsmath« 
\newtagform{simple}{}{}{}
\usetagform{simple}

\usepackage{tabularx} %schöne tabellen
\parindent0pt %einrücken verhindern

\usepackage{polynom}
% overall sans serif font
%\renewcommand{\familydefault}{\sfdefault}
\cfoot{\thepage  \ / \pageref{LastPage}}

% % % % % % % % % % % % % % % % % % % % % % % %
% % % % % % % % % % % % % % % % % % % % % % % %
\newcommand{\modullang}{AOT}
\newcommand{\modul}{AOT MSC}
\newcommand{\blatt}{01 Übungsblatt}
\newcommand{\tutorium}{Mittwoch 12:00 Uhr}
\newcommand{\tutor}{Dr. Fricke}
\newcommand{\datum}{17. November 2014}
\newcommand{\gruppe}{Gr05}
\newcommand{\RM}[1]{\MakeUppercase{\romannumeral #1}}
% % % % % % % % % % % % % % % % % % % % % % % %
% % % % % % % % % % % % % % % % % % % % % % % %

\begin{document} 

%%% Kopfzeile linker Bereich
%      gerade Seite   ungerade Seite
\lhead{\textbf{\modul}}
%%% Kopfzeile mittlerer Bereich
%      gerade Seite   ungerade Seite
\chead{\blatt}
%%% Kopfzeile linker Bereich
%      gerade Seite             ungerade Seite
\rhead{\gruppe}


	%-- Deckblatt --						      
	\title{\textbf{\modullang\\[0.25cm]}
		\normalsize{\blatt}} %Thema ändern	
	\author{\tutorium\\ \\
		xxx, 000000\\
		Tobias Pockrandt, 325550\\
		xxx, 000000\\ \\ \\ \\ \\ \\ \\ 
		Seitenzahl: \pageref{LastPage} \\ \\ \\ \\
		%Professor: Knipping\\
		Tutor: \tutor}
	\date{Abgabedatum: \datum} %Datum ändern
	\maketitle
	\newpage
	
	%\renewcommand \thesection {\arabic{section}.}
	%\renewcommand \thesubsection {\thesection. \arabic{subsection}}
	%\renewcommand \thesubsubsection {\thesubsection. \arabic{subsubsection}}
	

\renewcommand{\labelenumi}{\alph{enumi})}
\renewcommand{\labelenumii}{(\roman{enumii})}
\renewcommand{\labelenumiii}{\arabic{enumiii}.}
%\renewcommand{\labelenumii}{\textbf{-}}	
%-- Eigentlicher Text --
\section*{1. Aufgabe - Task Allokation per Auktion\hfill {\small (10 Punkte)}}
\begin{enumerate}
\item 
Für ein Spiel seien der Nutzen eines jeden Agenten $i$, falls er die Aution gewinnt, seine Wertschätzung $v_i$ minus sein Gebot $g_i$ und falls er nicht gewinnt 0.\\
$N(i) = \begin{cases}v_i - g_i&\text{falls gewonnen}\\0&\text{sonst}\end{cases} $ \\
Alle Spieler haben ein Maximalgebot von $v_i$ das sonst ihr Nutzen unter $0$ fallen könnte.\\
Nun ist ersichtlich, dass $1$ das höchste ($v_2$ das zweithöchste, etc.) Gebot setzen kann da $v_1$ echt größer als $v_{i\\1}$. Somit hat $1$ die Möglichkeit in jedem Spiel in dem er nicht Gewinnt auszuscheren und mehr zu bieten und damit höheren Nutzen zu erreichen.
Damit sind die Spiele mit dem Ausgang $(v_1 - v_2 + \epsilon$ $\epsilon \in [0,v_1-v_2]$ die einzigen Nash-Gleichgewichte, also diejenigen in denen $1$ die Wertschätzung von $v_2$ überbietet.

\item 
In Fall (2) ist, angenommen ein Gebot gewinnt, $g_{2.höchstes} \leq g_i \leq v_i$ der Nutzen $v_i - g_{2.höchstes}$. Für den Fall dass das erste und zweite Gebot gleich und der Index des Agenten kleiner als der des anderen ist, oder das Gebot verliert ist er 0. Der Agent hat also keine Chance auszuscheren um einen höheren Nutzen zu erzielen.
\item 
Unter keinen Umständen.
\end{enumerate}
\section*{2. Aufgabe - Dominanzelimination in gemischten Strategien\hfill {\small (5 Punkte)}}

\begin{center}
\begin{tabular}{c | cc}
	\RM{1}/\RM{2} & c & d \\ \hline
	A & 3,3 & 1,4\\
	B & 5,2 & -1,1\\
	C & 2,3 & 3,1\\
\end{tabular}
\end{center}

\begin{align*}
	Fall1: \RM{2} =  & c\\
	E_\RM{1}(A) = & 3\\
	E_\RM{1}(B) = & 5\\
	E_\RM{1}(C) = & 2\\
	\\
	Fall2: \RM{2} = & d\\
	E_\RM{1}(A) = & 1\\
	E_\RM{1}(B) = & -1\\
	E_\RM{1}(C) = & 3\\
\end{align*}
Behauptung: B und C Dominieren A\\
zu Fall1:
\begin{align}
	x_1,x_2 \in & [0,1]\\
	x_1 \cdot E_\RM{1}(B) + x_2 \cdot E_\RM{1}(C) \geq & E_\RM{1}(A)\\
	\Rightarrow & 5x_1 + 2x_2 \geq 3\\ 
	\text{Da Wahrscheinlichkeitsverteilung:  } & x_1 + x_2 = 1\\
	\Rightarrow & x_1 \geq \frac{1}{3}~~~~ x_2 \leq \frac{2}{3} 
\end{align}
zu Fall2:
\begin{align}
	x_1,x_2 \in & [0,1]\\
	x_1 \cdot E_\RM{1}(B) + x_2 \cdot E_\RM{1}(C) \geq & E_\RM{1}(A)\\
	\Rightarrow & -x_1 + 3x_2 \geq 1\\ 
	\text{Da Wahrscheinlichkeitsverteilung:  } & x_1 + x_2 = 1\\
	\Rightarrow & x_1 \leq \frac{1}{2}~~~~ x_2 \geq \frac{1}{2} 
\end{align}
Aus (5) und (10) folgt:  $\frac{1}{3} \leq x_1 \leq \frac{1}{2}$ und $x_2 = 1-x_1$ bzw. $\frac{1}{2} \leq x_2 \leq \frac{2}{3}$. Wobei $x_1 \equiv P(B)$ und $x_2 \equiv P(C) $. \\
Bei Verwendung dieser Wahrscheinlichkeitsverteilungen entfällt für Spieler \RM{1} die Option, sodass folgende verkleinerte Tabelle resultiert.\\

\begin{minipage}{0.2\textwidth}
	\begin{tabular}{c | cc}
		\RM{1}/\RM{2} & c & d \\ \hline
		A & 3,3 & 1,4\\
		B & 5,2 & -1,1\\
		C & 2,3 & 3,1\\
	\end{tabular}
\end{minipage}
$\Rightarrow$
\begin{minipage}{0.2\textwidth}
	\begin{tabular}{c | cc}
		\RM{1}/\RM{2} & c & d \\ \hline
		B & 5,2 & -1,1\\
		C & 2,3 & 3,1\\
	\end{tabular}
\end{minipage}
$\overset{(*)}{\Rightarrow}$
\begin{minipage}{0.2\textwidth}
	\begin{tabular}{c | c}
		\RM{1}/\RM{2} & c \\ \hline
		B & 5,2 \\
		C & 2,3 \\
	\end{tabular}
\end{minipage}
$\overset{(*^2)}{\Rightarrow}$
\begin{minipage}{0.2\textwidth}
	\begin{tabular}{c | c}
		\RM{1}/\RM{2} & c \\ \hline
		B & 5,2 \\
	\end{tabular}
\end{minipage}\\

(*): $E_\RM{2}(c) > E_\RM{2}(d)$ \\
(*): $E_\RM{1}(B) > E_\RM{1}(C)$\\
Somit ist \RM{1}(B) und \RM{2}(c) die dominante Strategie in gemischten Strategien.

\section*{3. Aufgabe - Trembling Hand Perfection\hfill {\small (5 Punkte)}}
\begin{tabular}{c | ccc}
	\RM{1}/\RM{2} & a & b & c \\ \hline
	A & \underline{1,1} & \underline{0},0 & -6,-4 \\
	\\
	B & 0,\underline{0} & \underline{0,0} & -4,\underline{-4} \\
	\\
	C & -4,-6 & \underline{-4},-4 & \underline{-4,-4} 
\end{tabular}\\

Reine Nash-Gleichgewichte sind demnach die Strategien $\{(A,a),(B,b),(C,c)\}$\\

\begin{align}
	E_\RM{1}(A) = & \frac{1}{3} \cdot 1 + \frac{1}{3} \cdot 0 + \frac{1}{3} \cdot -6 = \frac{1}{3} - \frac{6}{3} = -\frac{5}{3}\\
	E_\RM{1}(B) = & \frac{1}{3} \cdot 0 + \frac{1}{3} \cdot 0 + \frac{1}{3} \cdot -4 = - \frac{4}{3}\\
	E_\RM{1}(C) = & \frac{1}{3} \cdot -4 + \frac{1}{3} \cdot -4 + \frac{1}{3} \cdot -4 = - \frac{4}{3} - \frac{4}{3} - \frac{4}{3} = -\frac{12}{3} = -4\\
	& \text{da Tabelle Symertisch ist die Werte für spieler \RM{2} analog}\\
	E_\RM{2}(a) = & \frac{1}{3} \cdot 1 + \frac{1}{3} \cdot 0 + \frac{1}{3} \cdot -6 = \frac{1}{3} - \frac{6}{3} = -\frac{5}{3}\\
	E_\RM{2}(b) = & \frac{1}{3} \cdot 0 + \frac{1}{3} \cdot 0 + \frac{1}{3} \cdot -4 = - \frac{4}{3}\\
	E_\RM{2}(c) = & \frac{1}{3} \cdot -4 + \frac{1}{3} \cdot -4 + \frac{1}{3} \cdot -4 = - \frac{4}{3} - \frac{4}{3} - \frac{4}{3} = -\frac{12}{3} = -4
\end{align}

(A,a): (ist THP NGG)\\
Für Spieler \RM{1} sei $p_a = 1- \varepsilon_b - \varepsilon_c$ ($\varepsilon_b$ ist die Wkt für das Abweichen nach b; $\varepsilon_c$ ist die Wkt für das Abweichen nach c). Dann gilt:\\
\begin{align*}
E_\RM{1}(A) = & (1-\varepsilon_B-\varepsilon_c) - \varepsilon_c \cdot 6 = 1 - \varepsilon_b - 7\varepsilon_c\\
E_\RM{1}(B) = & - 4\varepsilon_c\\
E_\RM{1}(C) = & (1-\varepsilon_b-\varepsilon_c) \cdot -4 + \varepsilon_b \cdot -4 + \varepsilon_c \cdot -4 = -4
\end{align*}
Der Erwartungswert für Spieler \RM{1} ist größer in der Gleichgewichtsstrategie, wenn gilt:\\
$E_\RM{2}(A) > E_\RM{2}(B) \Leftrightarrow \varepsilon_b < 1 - 3\varepsilon_c$\\
$E_\RM{2}(A) > E_\RM{2}(C) \Leftrightarrow \varepsilon_b < 5 - 7\varepsilon_c$\\
Erfüllbar für beliebig kleine $\varepsilon_b$ und $\varepsilon_c$.\\
Da die Tabelle symmetrisch ist verläuft Spieler \RM{2} analog mit dem gleichem Ergebnis. \\
\\
(B,b): (ist THP NGG)\\
Für Spieler \RM{1} sei $p_b = 1- \varepsilon_a - \varepsilon_c$ ($\varepsilon_a$ ist die Wkt für das Abweichen nach a; $\varepsilon_c$ ist die Wkt für das Abweichen nach c). Dann gilt:\\
\begin{align*}
E_\RM{1}(A) = & \varepsilon_a - 6\varepsilon_c\\
E_\RM{1}(B) = & -4\varepsilon_c\\
E_\RM{1}(C) = & -4
\end{align*} 
Der Erwartungswert für Spieler \RM{1} ist größer in der Gleichgewichtsstrategie, wenn gilt:\\
$E_\RM{2}(B) > E_\RM{2}(A) \Leftrightarrow \varepsilon_c > \frac{1}{2}\varepsilon_a$\\
$E_\RM{2}(B) > E_\RM{2}(C) \Leftrightarrow \varepsilon_c < 1$\\
Erfüllbar für beliebig kleine $\varepsilon_a$ und $\varepsilon_c$\\
Da die Tabelle symmetrisch ist, verläuft Spieler \RM{2} analog mit dem gleichem Ergebnis.\\
\\
(C,c):\\
Für Spieler \RM{1} sei $p_c = 1- \varepsilon_a - \varepsilon_b$ ($\varepsilon_a$ ist die Wkt für das Abweichen nach a; $\varepsilon_b$ ist die Wkt für das Abweichen nach b). Dann gilt:\\
\begin{align*}
E_\RM{1}(A) = & \varepsilon_a - 6 (1-\varepsilon_a - \varepsilon_b) = -6 + 5\varepsilon_a + 6\varepsilon_b\\
E_\RM{1}(B) = & -4(1 - \varepsilon_a - \varepsilon_b) = -4 + 4\varepsilon_a + 4\varepsilon_b\\
E_\RM{1}(C) = & -4
\end{align*} 
Der Erwartungswert für Spieler \RM{1} ist größer in der Gleichgewichtsstrategie, wenn gilt:\\
$E_\RM{2}(C) > E_\RM{2}(A) \Leftrightarrow  \varepsilon_b < \frac{1}{3} - \frac{5}{6}\varepsilon_a$\\
$E_\RM{2}(C) > E_\RM{2}(B) \Leftrightarrow 0 > \varepsilon_a + \varepsilon_b$\\
Dies ist nicht möglich das $\varepsilon_a >= 0$ und $\varepsilon_b >= 0$ gelten muss. Daher kann (C,c) nicht THP sein.


\section*{4. Aufgabe - 2-Stufiges Ultimatumspiel mit Inflation\hfill {\small (5 Punkte)}}
a) / b)\\
\begin{landscape}
\begin{tikzpicture}[auto,node distance=2.2cm,thin]
	\node(1) {\RM{2}};
	\node(44) [below of=1] {\RM{1}};
	\node(53) [left of=44] {\RM{1}};
	\node(62) [left of=53] {\RM{1}};
	\node(71) [left of=62] {\RM{1}};
	\node(80) [left of=71] {\RM{1}};
	\node(35) [right of=44] {\RM{1}};
	\node(26) [right of=35] {\RM{1}};
	\node(17) [right of=26] {\RM{1}};
	\node(08) [right of=17] {\RM{1}};
	
	\node(J80) [below left=3cm and 0.5cm of 80] {8,0};
	\node(N80) [below right=3cm and 0.5cm of 80] {0,0};
	\node(J71) [below left=1.5cm and 0.5cm of 71] {7,1};
	\node(N71) [below right=1.5cm and 0.5cm of 71] {0,0};
	\node(J62) [below left=3cm and 0.5cm of 62] {6,2};
	\node(N62) [below right=3cm and 0.5cm of 62] {0,0};
	\node(J53) [below left=1.5cm and 0.5cm of 53] {5,3};
	\node(N53) [below right=1.5cm and 0.5cm of 53] {0,0};
	\node(J44) [below left=3cm and 0.5cm of 44] {4,4};
	\node(N44) [below right=3cm and 0.5cm of 44] {0,0};
	\node(J35) [below left=1.5cm and 0.5cm of 35] {3,5};
	\node(N35) [below right=1.5cm and 0.5cm of 35] {0,0};	
	\node(J26) [below left=3cm and 0.5cm of 26] {2,6};
	\node(N26) [below right=3cm and 0.5cm of 26] {0,0};
	\node(J17) [below left=1.5cm and 0.5cm of 17] {1,7};
	\node(N17) [below right=1.5cm and 0.5cm of 17] {0,0};
	\node(J08) [below left=3cm and 0.5cm of 08] {0,8};
	\node(N08) [below right=3cm and 0.5cm of 08] {0,0};
		
	\path[-,left] (1) edge node {4,4} (44);
	\path[-,bend right,left] (1) edge node {5,3} (53);
	\path[-,bend right,left] (1) edge node {6,2} (62);
	\path[-,bend right,left] (1) edge node {7,1} (71);
	\path[-,bend right,left] (1) edge node {8,0} (80);
	\path[-,bend left, right] (1) edge node {3,5} (35);
	\path[-,bend left, right] (1) edge node {2,6} (26);
	\path[-,red,bend left, right] (1) edge node {1,7} (17);
	\path[-,bend left, right] (1) edge node {0,8} (08);
	
	\path[-,red,right] (80) edge node {J} (J80);
	\path[-,left] (80) edge node {N} (N80);
	\path[-,red,left] (71) edge node {J} (J71);
	\path[-,right] (71) edge node {N} (N71);
	\path[-,red,right] (62) edge node {J} (J62);
	\path[-,left] (62) edge node {N} (N62);
	\path[-,red,left] (53) edge node {J} (J53);
	\path[-,right] (53) edge node {N} (N53);
	\path[-,red,right] (44) edge node {J} (J44);
	\path[-,left] (44) edge node {N} (N44);
	\path[-,red,left] (35) edge node {J} (J35);
	\path[-,right] (35) edge node {N} (N35);
	\path[-,red,right] (26) edge node {J} (J26);
	\path[-,left] (26) edge node {N} (N26);
	\path[-,red,right] (17) edge node {J} (J17);
	\path[-,left] (17) edge node {N} (N17);
	\path[-,right] (08) edge node {J} (J08);
	\path[-,red,left] (08) edge node {N} (N08);
\end{tikzpicture}
\end{landscape}
\begin{landscape}
		\begin{tikzpicture}[auto,node distance=2.2cm,thin]
		%[above right=0.7cm and 4cm of A]
		%Erstes Angebot
		\node(1) {\RM{1}};
		\node(55) [below of=1] {\RM{2}};
		\node(64) [left of=55] {\RM{2}};
		\node(73) [left of=64] {\RM{2}};
		\node(82) [left of=73] {\RM{2}};
		\node(91) [left of=82] {\RM{2}};
		\node(100) [left of=91] {\RM{2}};
		\node(46) [right of=55] {\RM{2}};
		\node(37) [right of=46] {\RM{2}};
		\node(28) [right of=37] {\RM{2}};
		\node(19) [right of=28] {\RM{2}};
		\node(010) [right of=19] {\RM{2}};
		
		%Neues Angebot
		\node(G100) [below=6cm of 100] {1,7};
		\node(G91) [below of=91] {1,7};
		\node(G82) [below=6cm of 82] {1,7};
		\node(G73) [below of=73] {1,7};
		\node(G64) [below=6cm of 64] {1,7};
		\node(G55) [below of=55] {1,7};
		\node(G46) [below=6cm of 46] {1,7};
		\node(G37) [below of=37] {1,7};
		\node(G28) [below=6cm of 28] {1,7};
		\node(G19) [below of=19] {1,7};
		\node(G010) [below=6cm  of 010] {1,7};
		
		%Jesnodes
		\node(J100) [left=0.5cm of G100] {10,0};
		\node(J91) [left=0.5cm of G91] {9,1};
		\node(J82) [left=0.5cm of G82] {8,2};
		\node(J73) [left=0.5cm of G73] {7,3};
		\node(J64) [left=0.5cm of G64] {6,4};
		\node(J55) [left=0.5cm of G55] {5,5};
		\node(J46) [left=0.5cm of G46] {4,6};
		\node(J37) [left=0.5cm of G37] {3,7};
		\node(J28) [left=0.5cm of G28] {2,8};
		\node(J19) [left=0.5cm of G19] {1,9};
		\node(J010) [left=0.5cm of G010] {0,10};
		
		%No-Nodes
		\node(N100) [right=0.5cm of G100] {0,0};
		\node(N91) [right=0.5cm of G91] {0,0};
		\node(N82) [right=0.5cm of G82] {0,0};
		\node(N73) [right=0.5cm of G73] {0,0};
		\node(N64) [right=0.5cm of G64] {0,0};
		\node(N55) [right=0.5cm of G55] {0,0};
		\node(N46) [right=0.5cm of G46] {0,0};
		\node(N37) [right=0.5cm of G37] {0,0};
		\node(N28) [right=0.5cm of G28] {0,0};
		\node(N19) [right=0.5cm of G19] {0,0};
		\node(N010) [right=0.5cm of G010] {0,0};
		
		\path[-,left] (1) edge node {5,5} (55);
		\path[-,bend right,left] (1) edge node {6,4} (64);
		\path[-,bend right,left] (1) edge node {7,3} (73);
		\path[-,bend right,left] (1) edge node {8,2} (82);
		\path[-,bend right,left] (1) edge node {9,1} (91);
		\path[-,bend right,above] (1) edge node {10,0} (100);
		\path[-,bend left, right] (1) edge node {4,6} (46);
		\path[-,bend left, right] (1) edge node {3,7} (37);
		\path[-,red,bend left, right] (1) edge node {2,8} (28);
		\path[-,bend left, right] (1) edge node {1,9} (19);
		\path[-,bend left, above] (1) edge node {0,10} (010);
		
		\path[-,red,left] (100) edge node {A} (G100);
		\path[-,red,left] (91) edge node {A} (G91);
		\path[-,red,left] (82) edge node {A} (G82);
		\path[-,red,left] (73) edge node {A} (G73);
		\path[-,red,left] (64) edge node {A} (G64);
		\path[-,red,left] (55) edge node {A} (G55);
		\path[-,red,left] (46) edge node {A} (G46);
		\path[-,red,left] (37) edge node {A} (G37);
		\path[-,left] (28) edge node {A} (G28);
		\path[-,left] (19) edge node {A} (G19);
		\path[-,left] (010) edge node {A} (G010);
		
		\path[-,left] (100) edge node {J} (J100);
		\path[-,left] (91) edge node {J} (J91);
		\path[-,left] (82) edge node {J} (J82);
		\path[-,left] (73) edge node {J} (J73);
		\path[-,left] (64) edge node {J} (J64);
		\path[-,left] (55) edge node {J} (J55);
		\path[-,left] (46) edge node {J} (J46);
		\path[-,left] (37) edge node {J} (J37);
		\path[-,red,left] (28) edge node {J} (J28);
		\path[-,red,left] (19) edge node {J} (J19);
		\path[-,red,left] (010) edge node {J} (J010);
		
		\path[-,right] (100) edge node {N} (N100);
		\path[-,right] (91) edge node {N} (N91);
		\path[-,right] (82) edge node {N} (N82);
		\path[-,right] (73) edge node {N} (N73);
		\path[-,right] (64) edge node {N} (N64);
		\path[-,right] (55) edge node {N} (N55);
		\path[-,right] (46) edge node {N} (N46);
		\path[-,right] (37) edge node {N} (N37);
		\path[-,right] (28) edge node {N} (N28);
		\path[-,right] (19) edge node {N} (N19);
		\path[-,right] (010) edge node {N} (N010);
		\end{tikzpicture}
\end{landscape}
c) Auf der geringsten Stufe in der \RM{1} 2 Münzen anbieten kann, werden sie sich beide auf (1,1) einigen (vgl. Baum unten). Mit diesem Ergebnis einigen sich beide auf der nächst höheren Stufe (\RM{2} bietet 4 Münzen) auf (2,2). Dies würde sich bis zu \RM{1} bietet 10 Münzen mit (5,5) fortsetzen.

\begin{tikzpicture}[auto,node distance=2.5cm,thin]
	\node(1) {\RM{1}};
	\node(11) [below of=1] {\RM{2}};
	\node(20) [left of=11] {\RM{2}};
	\node(02) [right of=11] {\RM{2}};
	
	\node(J20) [below left=3cm and 0.5cm of 20] {2,0};
	\node(N20) [below right=3cm and 0.5cm of 20] {0,0};
	\node(J11) [below left=1.5cm and 0.5cm of 11] {1,1};
	\node(N11) [below right=1.5cm and 0.5cm of 11] {0,0};
	\node(J02) [below left=3cm and 0.5cm of 02] {0,2};
	\node(N02) [below right=3cm and 0.5cm of 02] {0,0};
	
	\path[-,red,left] (1) edge node {1,1} (11);
	\path[-,bend right,left] (1) edge node {2,0} (20);
	\path[-,bend left,right] (1) edge node {0,2} (02);
	
	\path[-,right] (20) edge node {J} (J20);
	\path[-,red,left] (20) edge node {N} (N20);
	\path[-,red,left] (11) edge node {J} (J11);
	\path[-,right] (11) edge node {N} (N11);
	\path[-,red,right] (02) edge node {J} (J02);
	\path[-,left] (02) edge node {N} (N02);
\end{tikzpicture}

\section*{5. Aufgabe - WM-Spielort vergeben\hfill {\small (10 Punkte)}}

Die Angabe der Bestechungen ist wie folgt aufgebaut: $Xp_i$ wobei X die Menge ist die dem Entscheider $p_i$ geboten wird. 0all bedeutet das kein Entscheider bestochen wird.\\
Unter allem steht die Annahme das es Besser ist die WM auszutragen und mit 0-Nutzen hervorzugehen, als nicht zu Bestechen. \\
\begin{tabular}{c | cccccccccc}
	X$\setminus$ Y & 0all & 1$p_1$ & 1$p_2$ & 1$p_3$ & 1$p_1$1$p_2$ & 1$p_1$1$p_3$ & 1$p_2$1$p_3$ & 2$p_1$ & 2$p_2$ & 2$p_3$\\ \hline
	0all & 0,2 & 0,1 & 0,1 & 0,1 & 0,0 & 0,0 & 0,0 & 0,0 & 0,0 & 0,0\\
	1$p_1$ & -1,2 & -1,1 & -1,1 & -1,1 & -1,0 & -1,0 & -1,0 & -1,0 & -1,0 & -1,0\\
	1$p_2$ & -1,2 & -1,1 & -1,1 & -1,1 & -1,0 & -1,0 & -1,0 & -1,0 & -1,0 & -1,0\\
	1$p_3$ & -1,2 & -1,1 & -1,1 & -1,1 & -1,0 & -1,0 & -1,0 & -1,0 & -1,0 & -1,0\\
	1$p_1$1$p_2$ & 0,0 & -2,1 & -2,1 & 0,-1 & -2,0 & -2,0 & -2,0 & -2,0 & -2,0 & 0,-2\\
	1$p_1$1$p_3$ & 0,0 & -2,1 & 0,-1 & -2,1 & -2,0 & -2,0 & -2,0 & -2,0 & 0,-2 & -2,0\\
	1$p_2$1$p_3$ & 0,0 & 0,-1 & -2,1 & -2,1 & -2,0 & -2,0 & -2,0 & 0,-2 & -2,0 & -2,0\\
	2$p_1$& -2,2 & -2,1 & -2,1 & -2,1 & -2,0 & -2,0 & -2,0  & -2,0 & -2,0 & -2,0\\
	2$p_2$& -2,2 & -2,1 & -2,1 & -2,1 & -2,0 & -2,0 & -2,0  & -2,0 & -2,0 & -2,0\\
	2$p_3$& -2,2 & -2,1 & -2,1 & -2,1 & -2,0 & -2,0 & -2,0  & -2,0 & -2,0 & -2,0\\
\end{tabular}
\begin{center}
{\huge $\overset{Dominanz}{\Downarrow}$}
\end{center}
\begin{tabular}{c | cccccccccc}
	X$\setminus$ Y & 0all & 1$p_1$ & 1$p_2$ & 1$p_3$ & 1$p_1$1$p_2$ & 1$p_1$1$p_3$ & 1$p_2$1$p_3$ & 2$p_1$ & 2$p_2$ & 2$p_3$\\ \hline
	0all & 0,2 & 0,1 & 0,1 & 0,1 & 0,0 & 0,0 & 0,0 & 0,0 & 0,0 & 0,0\\
	1$p_1$1$p_2$ & 0,0 & -2,1 & -2,1 & 0,-1 & -2,0 & -2,0 & -2,0 & -2,0 & -2,0 & 0,-2\\
	1$p_1$1$p_3$ & 0,0 & -2,1 & 0,-1 & -2,1 & -2,0 & -2,0 & -2,0 & -2,0 & 0,-2 & -2,0\\
	1$p_2$1$p_3$ & 0,0 & 0,-1 & -2,1 & -2,1 & -2,0 & -2,0 & -2,0 & 0,-2 & -2,0 & -2,0\\
\end{tabular}
 \\
 \\
\begin{tabular}{c | cccccccccc}
	X$\setminus$ Y & 0all & 1$p_1$ & 1$p_2$ & 1$p_3$ & 1$p_1$1$p_2$ & 1$p_1$1$p_3$ & 1$p_2$1$p_3$ & 2$p_1$ & 2$p_2$ & 2$p_3$\\ \hline
	0all & \color{red}0,2 & 0,1 & 0,1 & 0,1 & 0,0 & 0,0 & 0,0 & 0,0 & 0,0 & 0,0\\
	1$p_1$1$p_2$ & 0,0 & \color{red}-2,1 & \color{red}-2,1 & 0,-1 & -2,0 & -2,0 & -2,0 & -2,0 & -2,0 & 0,-2\\
	1$p_1$1$p_3$ & 0,0 & \color{red}-2,1 & 0,-1 & \color{red}-2,1 & -2,0 & -2,0 & -2,0 & -2,0 & 0,-2 & -2,0\\
	1$p_2$1$p_3$ & 0,0 & 0,-1 & \color{red}-2,1 & \color{red}-2,1 & -2,0 & -2,0 & -2,0 & 0,-2 & -2,0 & -2,0\\
\end{tabular}
\begin{enumerate}
	\item In der letzten Tabelle sind die Effizientesten Antworten von Y auf die Bestechungen von X rot markiert. Daraus wird ersichtlich das Y immer mit 1 oder 2 Gewinnt. Für X ist das Beste Ergebnis keinen zu bestechen, somit bleibt X bei 0 und Y bei 2.
	\item Ohne die erweiterte Tabelle zu erstellen lassen sich Schlüsse ziehen. Das Bestechen eines einzelnen Entscheiders egal mit welcher Summe, kann nur Verlust generieren, da die beiden anderen sich somit für Y entscheiden. Jeweils 2 Entscheider mit jeweils einer Einheit zu bestechen führt ebenfalls zu keinem Gewinn. Einen mit 2 und einen mit 1 zu bestechen würde einen Verlust bedeuten, da Y nur den Entscheider mit der 1 bestechen müsste um zu gewinnen. Gleiches gilt für einen mit 3 und einen mit 1. Alle mit 1 Einheit zu bestechen, würde ebenfalls einen Verlust zu folge haben, da Y nur 2 Bestechen müsste und gewinnt. Zwei mit 1 und einen mit 2 zu Bestechen hätte das gleiche Ergebnis. Somit bleibt nur noch die Möglichkeit 2 Entscheider mit jeweils 2 Einheiten zu bestechen, jedoch reicht es Y einen mit 2 zu Bestechen und gewinnt Automatisch. \\
	Somit ist auch mit 4 Ressourcen für X kein Sieg möglich. Dies liegt daran das sich Y auf das Jeweilige Ergebnis einstellen kann und auch mittels der Gleichstandregel die Entscheidung des Komitees für sich entscheiden kann.
	\item 
	\begin{tabular}{c | c | c}
		Y $\setminus$ X & Optimale Antowort & Outcome\\ \hline
		0all & $1p_11p_2,1p_11p_3,1p_21p_3$ & 0,0~~X-Win\\ 
		1$p_1$ = 1$p_2$ = 1$p_3$ & Bestechung beider anderen mit einer Einheit & -1,0~~X-Win\\
		2$p_1$ = 2$p_2$ = 2$p_3$ & Bestechung beider anderen mit einer Einheit  & -2,0~~X-Win\\
		$1p_11p_2,1p_11p_3,1p_21p_3$ & 0all & 0,0~~Y-Win\\
	\end{tabular}\\
	Der Vorteil, das sich Y auf X einstellen kann ist entfallen, somit ist nur noch die Gleichheitsregel für Y. Dies führt nach Tabelle dazu, das Y im Besten Fall mit 0 aus der Bestechungsaffäre hervorgeht. Die Wahl steht lediglich zwischen X gewinnt mit 0,0 und Y gewinnt mit 0,0.
\end{enumerate}

\section*{6. Aufgabe - Konzentration auf Kernkompetenz\hfill {\small (10 Punkte)}}
\begin{enumerate}
	\item ~\\
	Ein Eroberter Markt bringt 6 und ein verlorener Markt -1, da beide im Markt seinen wollen. Ein Kampf ist für beide schlechter als das Verlassen des Marktes.\\ 
	\begin{tikzpicture}[auto,node distance=2cm,thin]
	\node(A) {A};
	\node(Aout) [below right of=A] {-1,6};
	\node(Ain) [below left of=A] {B};
	
	\node(Ba) [below left of=Ain] {-2,-2};
	\node(Bl) [below right of=Ain] {6,-1};
	
	\path[-,red,left] (A) edge node {In} (Ain);
	\path[-] (A) edge node {Out} (Aout);
	
	\path[-,left] (Ain) edge node {fight} (Ba);
	\path[-,red] (Ain) edge node {leave} (Bl);
	\end{tikzpicture}
	\item Nach dem Baum in Aufgabe a) würde A immer in den Markt gehen und B grundsätzlich verdrängen, weil B die Möglichkeit hat auf seinen Ersatzmarkt zurückzugreifen. Nimmt man B nun, wie aus der Aufgabenstellung, diese Rückzugsmöglichkeit würde es Grundsätzlich zum Kampf kommen und A würde nicht in den Markt einsteigen wollen.\\
	\begin{tikzpicture}[auto,node distance=2cm,thin]
	\node(A) {A};
	\node(Aout) [below right of=A] {-1,6};
	\node(Ain) [below left of=A] {B};
	
	\node(Ba) [below of=Ain] {-2,-2};
	
	\path[-,left] (A) edge node {In} (Ain);
	\path[-,red] (A) edge node {Out} (Aout);
	
	\path[-,red,left] (Ain) edge node {fight} (Ba);
	\end{tikzpicture}
\end{enumerate}
\end{document}